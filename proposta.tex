Aproximadamente 6,4 bilhões de coisas estarão conectadas a IoT até o final de 2016, e este número deverá chegar a 20,8 bilhões de objetos conectados em 2020 (seção \ref{sec:iot}). Faz-se observar diante deste cenário, que muitos objetos quando isolados, não têm como prover funcionalidade(s) que atendam ao requisito de um usuário final (outra coisa ou um humano). Então, para atender a tais requisitos, há necessidade de compor estes objetos com outros para oferecer nova(s) funcionalidade(s) que atendam aos usuários finais. Entretanto, a composição das coisas pode levar a uma interação de características a qual pode provocar efeitos colaterais indesejáveis (seção \ref{sec:featureinteraction}).

Diante desta visão da IoT, propõe-se um estudo de caso no HNS (seção \ref{sec:hns}), com o intuito de verificar se é possível detectar efeitos colaterais indesejáveis utilizando-se das redes neurais multilayer perceptron (seção \ref{subsec:neuralnetwork}) diante do cenário descrito na seção \ref{sec:cenario}. Para isto é realizado uma série de experimentos (seção \ref{ch:experimento}) dentro do cenário proposto a fim de se observar o comportamento dos dispositivos, obter dados para construção de um modelo de classificador baseado em multilayer perceptron, validar sua generalização e, caso a generalização seja confirmada através de métodos avaliativos \ref{subsec:evaluationMethods}, implantar o classificador no cenário proposto. Esta confirmação da generalização do modelo significa que é possível detectar efeitos colaterais indesejáveis utilizando-se das redes neurais multilayer perceptron para o cenário em questão.
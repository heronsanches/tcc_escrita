Solutions for specific feature interactions can then be generalized to other interactions of the same category.

The approaches described above require a-priori data about features and their potential interactions. Since this is a serious drawback with on-line approaches, approaches which circumvent that requirement have been developed. One way of achieving this is to introduce, during run-time, a separate phase of ”collecting” information before the actual operation of the approach. During this phase the behaviour of the feature is observed and the information is stored in a database. Aggoun and Combes [3] propose a “pre-deployment” phase where a passive observer gathers information about the feature behaviour in the network. The gathered information is then used by the active observer (essentially a feature manager) in the operation phase of the service to detect and resolve interactions. Similarly, Tsang and Magill [113] gather behaviour “signatures” of features in an iso- lated on-line environment (with just the feature under observation being active) and store those in a database. The feature manager then accesses this database during live network operation to detect and resolve interactions.

On-line techniques offer strengths which no other technique possesses. They address both detection and resolution. While this is not often the case with other approaches (e.g. formal methods), automated resolution is especially critical with on-line tech- niques. At design time, manually changing specifications is sufficient. However, feature interactions detected at run-time need to be resolved instantaneously to keep the in- tegrity of the network. On the other hand, work reported so far suggests that run-time resolution is difficult. This is due, in part,to the limited amount of information avail- able. Making an informed decision without much knowledge on the features is hard. Not surprisingly, approaches requiring a-priori knowledge offer the best resolutions. However, the biggest potential of on-line techniques lies in their ability to cope with additional services in the network at runtime and thus allowing for open and expand- able systems. A-priori knowledge is a major obstacle to use this potential. Thus, in terms of feature interaction resolution, many approaches opted for the obvious choice: terminating the call. However, is this a ”bad” resolution? A single call is very cheap. Thus terminating a call might at most annoy the affected users. On the other hand a complex algorithm finding a good resolution is possibly very expensive in terms of its run-time behaviour. Further, the resolution might still confuse the users and mismatch their expectations. To progress work on this issue work is ongoing to augment run-time approaches with feature independent information from off-line techniques to guide the resolution. These approaches are referred to as hybrids. There is very little published work on hybrid approaches to date, however, two examples are the work by Calder et al. [23,105] and by Aggoun and Combes [3].
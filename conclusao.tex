O presente trabalho demonstrou através de um cenário (``Levar compras'' - Seção \ref{sec:cenario}) no HNS que quando dispositivos são compostos para prover uma única funcionalidade, o comportamento de pelo menos um destes pode provocar interação de características com efeitos colaterais indesejáveis, no caso deste cenário, provocados por violação de hipótese entre o HS e o elevador e, entre o elevador e a garra. Os dispositivos do cenário ``Levar compras'' são disponibilizados como serviços Web RESTFul, então estes têm capacidade de se comunicar, ser identificado unicamente e oferecer serviços de forma independente, representando desta forma um cenário da IoT.

Para o cenário em questão foi proposto detectar tais efeitos colaterais indesejáveis, de forma online, utilizando-se de um \textit{Feature Interaction Manager} (FIM). Para que o FIM pudesse realizar a detecção dos efeitos colaterais indesejáveis foi implantado um modelo do \textit{ensemble} (DECORATE) na funcionalidade ``Levar compras'', este atua justamente no momento em que o FIM pede informações de restrição e contéudo da cesta que está sobre o carro. O FIM então pega as informações do conteúdo da cesta, transforma-as em formato de dados que o DECORATE entenda (neste caso em específico, um arquivo arff) e realiza a classificação em ``é efeito colateral'' ou ``não é efeito colateral''. 

O modelo implantado no FIM foi concebido através de um processo classificatório (Figura \ref{fig:slclassification}), no qual as etapas podem ser representadas em (\ref{sec:cenario}, \ref{sec:selecaodados}, \ref{ch:validacao}). Na etapa de construção, avaliação e validação (Capítulo \ref{ch:validacao}) utilizou-se o método \textit{stratified-10-fold-crossvalidation} repetido dez vezes diante diferentes porções do \textit{dataset}, com o intuito de saber a partir de quantas instâncias o modelo iria generalizar. Ficou observado que o melhor resultado obtido foi quando o modelo construído utilizou-se de 45 instâncias do total de 59 instâncias do \textit{dataset}. Então para construir o modelo final, o qual foi  implantado no FIM, executou-se o \textit{stratified-10-fold-crossvalidation} (repetido 10 vezes) por diversas vezes sobre 45 instâncias do \textit{dataset} (escolhidas aleatoriamente com seeds diferentes) até que a expressão \ref{eq:esxpvalclass} fosse satisfeita, esta é baseada nos resultados obtidos da Tabela \ref{table:valorescurva} com dataset de tamanho 45. O resultado da expressão satisfeita gerou um modelo classificatório final com $\textit{Accuracy}=98.05\%$, $\textit{Precision}=98.5\%$ e $\textit{Recall}=97.67\%$.

Assim, o FIM ficou apto a realizar detecção inteligente de efeitos colaterais indesejáveis para o cenário proposto, mostrando que é possível detectar efeitos colaterais indesejáveis entre dispositivos na visão da IoT.

Como trabalhos futuros, pretende-se:
\begin{itemize}
\item Realizar um estudo sobre algoritmos de seleção de atributos afim de não mais selecionar um vetor de atributos (que descreve o problema) de forma manual. Observe que os atributos selecionados de forma manual e visual (Seção \ref{sec:selecaodados}) para o cenário proposto neste trabalho estavam em um espaço de 2 dimensões (Figura \ref{fig:plotall1}), algumas vezes um espaço de duas dimensões não é suficiente para descrever um problema, o mesmo acontece para um espaço 3D (o máximo que a visão humana consegue visualizar). Realizado este estudo, então pretende-se executar os experimentos realizados neste trabalho (Capítulo \ref{ch:validacao}) em cima dos vetores de atributos selecionados pelos algoritmos estudados e realizar comparações dos resultados obtidos com cada um dos algoritmos de seleção de atributos utilizados.
\item Criar um cenário, com mais dispositivos, que seja o mais real possível e com pouca intervenção humana ou nenhuma e utilizar os algoritmos de seleção de atributos estudados e realizar as mesmas comparações.
\item Estudar sobre aprendizado não supervisionado e prover metodologia para realizar classificação de efeitos colaterais indesejáveis no cenário deste trabalho e no outro que pretende-se criar citado anteriormente.
\end{itemize}

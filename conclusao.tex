O presente trabalho demonstrou através de um cenário (``Levar compras'' - seção \ref{sec:cenario}) no HNS que quando dispositivos são compostos para prover uma única funcionalidade, o comportamento de pelo menos um destes pode provocar interação de características com efeitos colaterais indesejáveis. Foi proposto detectar tais efeitos colaterais indesejáveis diante do cenário apresentado utilizando-se das mensagens trocadas entres os dispositivos, mais especificamente as informações dos objetos que estavam a ser transportados pelo veículo. A partir dessas informações foi realizado um processo classificatório (seções \ref{subsec:obsproblema}, \ref{sec:selecaodados}, \ref{sec:constvalidacao}) a fim de se construir um modelo final de classificador que tivesse alto grau de generalização para o problema de detectar efeitos colaterais indesejáveis no cenário em questão. Por fim, tal modelo foi construído e implantado no componente de software, FIM, do HS, mostrando que é possível detectar efeitos colaterais indesejáveis entre dispositivos no cenário da IoT.


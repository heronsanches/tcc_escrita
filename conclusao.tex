O presente trabalho demonstrou através de um cenário (``Levar compras'' - seção \ref{sec:cenario}) no HNS que quando dispositivos são compostos para prover uma única funcionalidade, o comportamento de pelo menos um destes pode provocar interação de características com efeitos colaterais indesejáveis. Então, para o cenário em questão foi proposto detectar tais efeitos colaterais indesejáveis, de forma online, utilizando-se de um \textit{Feature Interaction Manager} (FIM). Para que o FIM pudesse realizar a detecção dos efeitos colaterais indesejáveis foi implantado um modelo de meta-classificador (DECORATE) na funcionalidade ``Levar compras''.

O modelo implantado no FIM foi concebido através de um processo classificatório (Figura \ref{fig:slclassification}), no qual as etapas podem ser representadas pelas seções (\ref{sec:cenario}, \ref{sec:selecaodados}, \ref{sec:constvalidacao}). Na etapa de validação utilizou-se o método \textit{stratified-10-fold-crossvalidation} repetido dez vezes diante diferentes porções do \textit{dataset}, com o intuito de saber a partir de quantas instâncias o modelo iria generalizar. Ficou observado que o melhor resultado obtido foi quando o modelo construído utilizou-se de 45 instâncias do total de 59 instâncias do \textit{dataset}. Então para construir o modelo final, o qual foi  implantado no FIM, executou-se o \textit{stratified-10-fold-crossvalidation} (repetido 10 vezes) por diversas vezes sobre 45 instâncias do \textit{dataset} (escolhidas aleatoriamente com seeds diferentes) até que a expressão \ref{eq:esxpvalclass} fosse satisfeita, a qual foi composta baseada no melhor resultado obtido anteriormente (o com 45 instâncias). O resultado da expressão satisfeita gerou um modelo classificatório final com $\textit{Accuracy}=98.05\%$, $\textit{Precision}=98.5\%$ e $\textit{Recall}=97.67\%$, como pode ser visualizado na Tabela \ref{table:valmodelfinal}.

Desta forma o FIM ficou apto a realizar detecção inteligente de efeitos colaterais indesejáveis para o cenário proposto, mostrando que é possível detectar efeitos colaterais indesejáveis entre dispositivos na visão da IoT.

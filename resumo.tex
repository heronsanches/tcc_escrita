Diversos dispositivos têm sido incorporados em ambientes domésticos através dos \textit{Home Server} (HS) com o intuito de automatizar as tarefas cotidianas de uma casa inteligente. Estes dispositivos podem ser disponibilizados como serviços Web RESTFul. Porém, muitos destes dispositivos independentes não conseguem atender aos requisitos dos usuários. Assim, há a necessidade de composição destes com outros dispositivos para prover funcionalidades que atendam a esses requisitos. Quando existe uma composição de dispositivos, o comportamento de pelo menos um destes pode provocar interação de características com efeitos colaterais indesejáveis. Assim, o presente trabalho propõe detectar efeitos colaterais indesejáveis entre dispositivos utilizando-se meta-classificação. Para isto, observou-se o comportamento físico e as mensagens trocadas entre os dispositivos. Através do conjunto de dados obtidos realizou-se um processo classificatório e como resultado final foi obtido um modelo com mais de 90\% de precisão, o qual foi incorporado no FIM (\textit{Feature Interaction Manager}) do \textit{Home Network System} (HNS), mostrando que é possı́vel detectar efeitos colaterais indesejáveis entre dispositivos.

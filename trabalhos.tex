\cite{Maternaghan:2013} propõem uma abordagem \textit{offline} (na fase de construção do sistema) para detecção de efeitos colaterais indesejáveis causados por conflitos de políticas. Nesta abordagem é verificado as ações de cada política de forma isolada e os pares de políticas que causam conflitos.

Em \cite{Wilson:2008} é apresentado um modelo de gerenciamento de Unidade de Funcionalidade de Software (UFS) ou \textit{feature} em um \textit{Home Automation System} para detecção de interação de características. Caso um efeito colateral indesejado ocorra, este deve ser evitado, mas se ocorre um efeito colateral desejado, este deve ser permitido. Este modelo consiste em três camadas de cima para baixo, como segue.
\begin{itemize}
   \item Camada superior (serviços que automatizam a casa) - utiliza um ou mais dispositivos, os quais estão localizados na camada intermediária (dispositivos);
   \item Camada de dispositivos – contem dois tipos de dispositivos (entrada (ex.: termômetro, somente monitora um aspecto do ambiente e retorna para os serviços) e saída (ex.: aquecedor, o qual altera o ambiente))
   \item Camada de ambiente – contêm variáveis de ambiente, por exemplo: movimento do quarto, temperatura do quarto, luminosidade do quarto, umidade do quarto, fumaça do quarto.
\end{itemize} O gerenciamento de UFS trabalha controlando o acesso à camada de ambiente utilizando de variáveis de bloqueio de acesso, especificando prioridades de serviços e além disso se utiliza de um banco de dados na nuvem para manter detalhes a respeito de cada dispositivo, com a finalidade de entender o efeito das ações de cada dispositivo no ambiente.

Os autores em \cite{Nakamura:2009} apresentam um método de detecção e resolução online para efeitos colaterais indesejáveis entre serviços integrados no \textit{Home Network System}. Os autores definem dois tipos de efeitos colaterais indesejáveis, \textit{appliance interaction} e \textit{environment interaction}. \textit{Appliance interaction} refere-se a uma situação onde dois serviços compartilham um mesmo dispositivo de forma conflitante, enquanto que \textit{environment interaction} ocorre quando dois serviços, os quais não necessariamente compartilham os mesmos dispositivos, mas conflitam-se por propriedades de ambientes, como, por exemplo, luminosidade. O método proposto utiliza mecanismos de suspender/resumir, prioridade e tempo de ativação dos serviços para detectar e resolver as interações. Em outro trabalho mais adiante \cite{Ikegami:2013}, no qual Nakamura e Ikegami participaram, estes com Matsumoto propuseram uma extensão do trabalho anterior. Nesta, os autores definem um modelo (\textit{environment impact model}) o qual define como cada dispositivo contribui para as variáveis de ambiente. Também é introduzido um \textit{environment requirement} para definir o estado esperado de cada serviço. Então  o conceito anterior de environment interaction é reformalizado introduzindo uma condição que uma certa quantidade de impactos acumulados viola os requerimentos dos serviços em questão. O autor realizou cinco casos de estudos os quais conseguiu detectar interações que não tinha conseguido com a definição anterior.

\cite{Alfakeeh:2016} utilizam um método de detecção e resolução online o qual é um mecanismo de negociação entre os serviços ou as preferências do usuário da casa a fim de que seja possível estes serviços trabalharem em conjunto ao mesmo tempo. Este mecanismo de negociação é um \textit{Agent-Based Negotiation System} (ABNS) de detecção e resolução de efeitos colaterais indesejáveis na casa inteligente.

A proposta apresentada neste TCC apresenta um método de detecção online, assim como em \cite{Wilson:2008, Nakamura:2009, Ikegami:2013, Alfakeeh:2016}. Nos trabalhos citados, na arquitetura de HNS utilizada, os dispositivos são utilizados para prover serviços ao usuário da casa, e estes são acessados somente pelo HS através das APIs dos dispositivos. Os dispositivos nesta arquitetura não têm capacidade de se comunicar e oferecer serviços entre eles de forma independente. No trabalho desenvolvido neste TCC, os dispositivos são disponibilizados como serviços Web RESTFul e estes têm capacidade de se comunicar e oferecer serviços de forma independente, representando desta forma um cenário da IoT. Além disto, o método de detecção online proposto no Capítulo \ref{ch:proposta} utiliza aprendizagem de máquina para detecção dos efeitos colaterais indesejáveis, mais precisamente se utilizou do \textit{ensemble} DECORATE para prover tal capacidade.

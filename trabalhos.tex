\cite{Wilson:2008} 

Os autores em \cite{Nakamura:2009} apresentam um método de detecção e resolução online para interação de características entre serviços integrados no \textit{Home Network System}. Os atores definem dois tipos de interação de características, \textit{appliance interaction} e \textit{environment interaction}. \textit{Appliance interaction} refere-se a uma situação onde dois serviços compartilham um mesmo dispositivo de forma conflitante, enquanto que \textit{environment interaction} ocorre quando dois serviços, os quais não necessariamente compartilham os mesmos dispositivos, mas conflitam-se por propriedades de ambientes, como por exemplo luminosidade. O método proposto utiliza mecanismos de suspender/resumir, prioridade e tempo de ativação dos serviços para detectar e resolver as interações. Na arquitetura de HNS utilizada, os dispositivos são utilizados para prover serviços ao usuário da casa, e estes são acessados somente pelo HS através das APIs dos dispositivos. Os dispositivos nesta arquitetura não têm capacidade de se comunicar e oferecer serviços entre eles de forma independente. Em outro trabalho mais adiante \cite{Ikegami:2013} deste, no qual Nakamura e Ikegami participaram, estes juntamente com Matsumoto propuseram uma extensão do trabalho anterior. Nesta, os autores definem um modelo (\textit{environment impact model}) o qual define como cada dispositivo contribui para as variáveis de ambiente. Também é introduzido um \textit{environment requirement} para definir o estado esperado de cada serviço. Então  o conceito anterior de environment interaction é reformalizado introduzindo uma condição que uma certa quantidade de impactos acumulados viola os requerimentos dos serviços em questão. O autor realizou cinco casos de estudos os quais conseguiu detectar interações que não tinha conseguido com a definição enterior.

\cite{Maternaghan:2013}
\cite{Alfakeeh:2016}


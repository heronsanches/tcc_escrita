A Internet nesta última década tem contribuído de forma significativa na economia e sociedade, deixando como legado uma sólida infraestrutura de rede de comunicação. O seu maior disseminador nesse período vem sendo \textit{World Wide Web}(WWW), o qual permite o compartilhamento de informação e mídia de forma global \cite{Chandrakanth:2014}.

A Internet está se tornando cada vez mais presente no cotidiano, devido, por exemplo, ao crescente número de usuários de dispositivos móveis, os quais possuem tecnologias de conexão com a Internet, as quais cada dia tornam-se mais acessíveis \cite{Chandrakanth:2014}.

Em 2010 havia aproximadamente 1,5 bilhão de PCs conectados à Internet e mais que 1 bilhão de telefones móveis \cite{Sundmaeker:2010}. Segundo Gartner\footnotemark \footnotetext{\url{http://www.gartner.com/newsroom/id/3165317}}, 6,4 bilhões de coisas estarão conectadas até o final de 2016 e, de acordo com estimativas da CISCO\footnotemark \footnotetext{\url{http://www.cisco.com/c/en/us/solutions/collateral/enterprise/cisco-on-cisco/Cisco_IT_Trends_IoE_Is_the_New_Economy.html}}, em 2020 esse número atingirá cerca de 50 bilhões. A previsão de \cite{Sundmaeker:2010}, a qual dizia que a denominada Internet dos PCs seria movida para o que se chama de Internet das Coisas fica então mais evidente neste atual cenário.

\textit{Internet of Things} (IoT), traduzido para o português como Internet das Coisas, pode ser definida como uma rede de coisas que têm a capacidade de sentir, identificar e compreender o ambiente em que estão imersas \cite{Ashton:2009}. ``Coisas ou objetos'', tais como, RFID tags, sensores, telefones móveis, dentre outros, através de esquemas de endereçamento único são capazes de interagir com os outros e cooperar com seus vizinhos para alcançar um objetivo em comum \cite{Atzori:2010}.

Uma definição tecnológica para IoT pode ser como em \cite{Sundmaeker:2010}, o qual diz que IoT é parte integrante da futura Internet e pode ser definida como uma infraestrutura de rede global dinâmica com capacidades de autoconfiguração baseada nos padrões e interoperabilidade dos protocolos de comunicação onde coisas físicas e virtuais têm identidade, atributos físicos, personalidade virtual, usam interfaces inteligentes e, são integradas dentro da rede de informações.

Apesar da grande potencialidade da Internet das Coisas, a qual gerou em 2015 um faturamento em torno de 130,33 bilhões de dólares americanos e tem prospecção de chegar até 883.55 bilhões em 2020\footnotemark \footnotetext{http://www.marketsandmarkets.com/Market-Reports/iot-application-technology-market-258239167.html}, ainda existem muitos desafios a serem vencidos, tais como segurança da informação \cite{Roman:2013}, armazenamento e processamento de grande quantidade de dados \cite{Zaslavsky:2013}, dentre outros. Adentrando em um dos desafios da IoT, o da disponibilidade de uma interface de comunicação e programação comum aos objetos, pode-se disponibilizar os dispositivos como serviços Web RESTFull, desta forma pode-se utilizar os protocolos Web como linguagem comum de integração dos dispositivos à Internet \cite{Franca:2011, Roy:2015, Mineraud:2016}.

Segundo \cite{Heffelfinger:2014}, serviços Web RESTFull são serviços que seguem os princípios arquiteturais REST (Representational State Transfer), onde os serviços Web são vistos como recursos e podem ser identificados por Uniform Resources Identifiers (URIs). 

Os principais princípios do estilo arquitetural REST são: endereçamento global através de identificação dos recursos, interface uniforme compartilhada por todos os recursos, interações \textit{stateless} entre os serviços e, mensagens auto-descritivas e \textit{hypermedia} como um mecanismo para descoberta descentralizada de recursos por referência \cite{Pautasso:2014}.

Diante deste cenário onde diversos dispositivos são combinados para alcançar um objetivo em comum \cite{Kranz:2010, Atzori:2010, Whitmore:2015} e pelo fato destes comumente serem disponibilizados como serviços Web, e da crescente adição destes à IoT, surge a necessidade de tratar um outro problema ainda não citado, que ocorre quando a composição de dispositivos (os quais oferecem alguma funcionalidade) leva a um comportamento não esperado (interação de características) nesta composição, o qual pode resultar em um estado inconsistente, instabilidade ou dados imprecisos - efeito colateral indesejável \cite{NHLABATSI:2008}.

Dentre as possíveis causas \cite{Weiss:2007} de efeitos colaterais indesejáveis pode-se citar:

\begin{itemize}
\item \textit{Goal Conflit} (Conflito de interesses): Cada característica ou unidade de adicional funcionalidade ao software (funcionalidade) tem seus próprios objetivos (designadas para fazer algum tipo de processamento, coletar algum dado, dar uma saída específica, dentre outros) a serem alcançados. Entretanto, quando as características são combinadas, os objetivos das duas características podem entrar em conflito e não se pode garantir que todos sejam alcançados. 
\item \textit{Violation of Assumptions} (Violação de suposições): Desenvolvedores das funcionalidades de software precisam fazer algumas suposições de como outras características trabalham. Estes podem fazer suposições incorretas, por exemplo, devido a semântica ambígua (tal como uso do mesmo conceito de diferentes formas), ou devido a presença de diferentes versões da mesma funcionalidade de software. De forma similar, implementação de características podem ser baseadas em suposições incorretas sobre o contexto de uso. Uma das características de uma \textit{Violation of Assumptions} é quando a mudança em uma \textit{feature} quebra a suposição correta que a outra característica tinha sobre esta.
\item \textit{Policy Conflict} (Conflito de políticas): Políticas proveem os meios para especificação e modularização do comportamento de uma característica, na medida em que alinha suas capacidades e restrições com os requisitos de seus usuários. A \textit{Policy Conflict} ocorre caso exista políticas (autenticação, ou de privacidade) especificadas em duas \textit{features} que se referem as suas correspondentes operações, e que as políticas não são compatíveis.
\end{itemize}

O presente trabalho propõe:
\begin{itemize}
\item Criar um cenário (``Levar compras'') que simule efeitos colaterais indesejáveis diante da visão da IoT, os quais são causados por violação de hipótese. Para isto, é realizado um estudo de caso em um \textit{Home Network System} (HNS). HNS é uma rede doméstica de coisas (aparelhos domésticos e sensores) com capacidade de conectividade de rede e, interface de controle e monitoramento. Nesse tipo de sistema os aparelhos e sensores podem juntos prover novas funcionalidades, as quais atendam as expectativas do usuário da casa. O HNS tem um componente denominado de \textit{Home Server} (HS), este acessa os dispositivos através de APIs e disponibiliza as funcionalidades ao usuário final \cite{Nakamura:2009, Ikegami:2013}. Os dispositivos no cenário ``Levar compras'' são serviços independentes disponibilizados como serviços Web RESTFul.
\item Detectar efeitos colaterais indesejáveis de forma inteligente no cenário ``Levar compras'' utilizando o método \textit{ensemble} DECORATE. Este tipo de método utiliza um conjunto de classificadores para construir diversas hipóteses (modelos). Então é utilizado o conhecimento de cada modelo afim de se chegar em uma decisão final mais confiável \cite{Melville:2004}. Diversas propostas de detecção de efeitos colaterais indesejáveis têm sido abordadas em HNS \cite{Wilson:2008, Nakamura:2009, Ikegami:2013, Maternaghan:2013, Alfakeeh:2016}, nenhuma destas propostas citadas utiliza alguma metodologia de detecção que utilize classificadores para detecção de efeitos colaterais indesejáveis. Na arquitetura de HNS utilizada nestes trabalhos, os dispositivos são utilizados para prover serviços ao usuário da casa, e estes são acessados somente pelo HS através das APIs dos dispositivos. Os dispositivos nesta arquitetura não têm capacidade de se comunicar e oferecer serviços entre eles de forma independente. No trabalho desenvolvido neste TCC, os dispositivos (elevador, garra e veículo terrestre) são disponibilizados como serviços Web RESTFul e estes têm capacidade de se comunicar e oferecer serviços de forma independente, representando desta forma um cenário da IoT.
\end{itemize}

O cenário ``Levar compras'' é a composição dos três dispositivos independentes citados (elevador, garra e veículo terrestre) os quais juntos compõem a funcionalidade ``Levar compras'', que é disponibilizada ao usuário humano do HNS através do HS (que também é um serviço Web RESTFul).

Através do \textit{ensemble} DECORATE foi possível detectar os efeitos colaterais indesejáveis com mais de 90\% de precisão. 

O restante deste trabalho está organizado da seguinte maneira:

\begin{itemize}
\item Capítulo \ref{ch:theory}: Apresenta todo embasamento teórico necessário para compreender o estudo realizado neste trabalho.
\item Capítulo \ref{ch:proposta}: Esse capítulo descreve a proposta para detecção de efeitos colaterais indesejáveis no cenário ``Levar compras''.
\item Capítulo \ref{ch:validacao}: Trata dos procedimentos realizados para validar a proposta de pesquisa deste TCC.
\item Capítulo \ref{ch:trabalhos}: Descreve de forma resumida os trabalhos relacionados ao desta monografia, assim como em que difere destes.
\item Capítulo \ref{ch:conclusion}: Este capítulo é uma síntese da investigação e dos experimentos realizados neste TCC.
\end{itemize}

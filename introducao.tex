A Internet nesta última década tem contribuído de forma significativa na economia e sociedade, deixando como legado uma sólida infraestrutura de rede de comunicação. O seu maior disseminador nesse período vem sendo \textit{World Wide Web}(WWW), o qual permite o compartilhamento de informação e mídia de forma global \cite{Chandrakanth:2014}.

A Internet está se tornando cada vez mais presente no cotidiano, devido, por exemplo, ao crescente número de usuários de dispositivos móveis, os quais possuem tecnologias de conexão com a Internet, as quais cada dia tornam-se mais acessíveis \cite{Chandrakanth:2014}.

Em 2010 havia aproximadamente 1,5 bilhão de PCs conectados à Internet e mais que 1 bilhão de telefones móveis \cite{Sundmaeker:2010}. Segundo Gartner\footnotemark \footnotetext{\url{http://www.gartner.com/newsroom/id/3165317}}, 6,4 bilhões de coisas estarão conectadas até o final de 2016 e, de acordo com estimativas da CISCO\footnotemark \footnotetext{\url{http://www.cisco.com/c/en/us/solutions/collateral/enterprise/cisco-on-cisco/Cisco_IT_Trends_IoE_Is_the_New_Economy.html}}, em 2020 esse número atingirá cerca de 50 bilhões. A previsão de \cite{Sundmaeker:2010}, a qual dizia que a denominada Internet dos PCs seria movida para o que se chama de Internet das Coisas fica então mais evidente neste atual cenário.

\textit{Internet of Things} (IoT), traduzido para o português como Internet das Coisas, pode ser definida como uma rede de coisas que têm a capacidade de sentir, identificar e compreender o ambiente em que estão imersas \cite{Ashton:2009}. ``Coisas ou objetos'', tais como, RFID tags, sensores, telefones móveis, dentre outros, através de esquemas de endereçamento único são capazes de interagir com os outros e cooperar com seus vizinhos para alcançar um objetivo em comum \cite{Atzori:2010}.

Uma definição tecnológica para IoT pode ser como em \cite{Sundmaeker:2010}, o qual diz que IoT é parte integrante da futura Internet e pode ser definida como uma infraestrutura de rede global dinâmica com capacidades de autoconfiguração baseada nos padrões e interoperabilidade dos protocolos de comunicação onde coisas físicas e virtuais têm identidade, atributos físicos, personalidade virtual, usam interfaces inteligentes e, são integradas dentro da rede de informações.

Apesar da grande potencialidade da Internet das Coisas, a qual gerou em 2015 um faturamento em torno de 130,33 bilhões de dólares americanos e tem prospecção de chegar até 883.55 bilhões em 2020\footnotemark \footnotetext{http://www.marketsandmarkets.com/Market-Reports/iot-application-technology-market-258239167.html}, ainda existem muitos desafios a serem vencidos, tais como segurança da informação \cite{Roman:2013}, armazenamento e processamento de grande quantidade de dados \cite{Zaslavsky:2013}, dentre outros. Adentrando em um dos dessafios da IoT, o da disponibilidade de uma interface de comunicação e programação comum aos objetos, pode-se disponibilizar os dispositivos como serviços Web RESTFull, desta forma pode-se utilizar os protocolos Web como linguagem comum de integração dos dispositivos à Internet \cite{Franca:2011, Mineraud:2016}.

Diante deste cenário onde diversos dispositivos são combinados para alcançar um objetivo em comum \cite{Kranz:2010, Atzori:2010, Whitmore:2015}, e da crescente adição destes à IoT, surge a necessidade de tratar um outro problema ainda não citado, que ocorre quando a composição de dispositivos (os quais oferecem alguma funcionalidade) leva a um comportamento não esperado (interação de características) por esta composição, o qual pode resultar em um estado inconsistente, instabilidade ou dados imprecisos - efeito colateral indesejável \cite{NHLABATSI:2008}.

**falar de algumas posíveis causas**

**falr tudo o que o trabalho propoe**
Assim, o presente trabalho propõe detectar efeitos colaterais indesejáveis entre dispositivos através da meta-classificação, metodologia que utiliza um conjunto de classificadores para construir diversas hipóteses (modelos). Então, é utilizado o conhecimento de cada modelo afim de se chegar em uma decisão final mais confiável \cite{Melville:2004}.

Os dados de treino foram gerados manualmente a partir de um conjunto de objetos do cenário ``Levar compras'', o qual foi utilizado no estudo de caso do \textit{Home Network System}.

Os resultados obtidos demonstraram que através da meta-classificação foi possível detectar os efeitos colaterais indesejáveis com mais de 90\% de precisão.

O restante deste trabalho está organizado da seguinte maneira:

\begin{itemize}
\item Capítulo \ref{ch:theory}: Apresenta todo embasamento teórico necessário para compreender o estudo realizado neste trabalho.
\item Capítulo \ref{ch:proposta}: Esse capítulo descreve a proposta para detecção de efeitos colaterais indesejáveis no cenário ``Levar compras''.
\item Capítulo \ref{ch:validacao}: Trata dos procedimentos realizados para validar a proposta de pesquisa deste TCC.
\item Capítulo \ref{ch:conclusion}: Este capítulo é uma síntese da investigação e dos experimentos realizados neste TCC.
\end{itemize}

A Internet nesta última década tem contribuído de forma significativa na economia e sociedade, deixando como legado uma notável infraestrutura de rede de comunicação. O seu maior disseminador nesse período vem sendo \textit{World Wide Web}(WWW), o qual permite o compartilhamento de informação e mídia de forma global \cite{Chandrakanth:2014}.

A Internet está se tornando cada vez mais persistente no cotidiano, devido, por exemplo, ao crescente número de usuários de dispositivos móveis, os quais possuem tecnologias de conexão com a Internet, as quais cada dia tornam-se mais acessíveis \cite{Chandrakanth:2014}.

Em 2010 havia aproximadamente 1,5 bilhão de PCs conectados a Internet e mais que 1 bilhão de telefones móveis \cite{Sundmaeker:2010}. Segundo Gartner\footnotemark \footnotetext{\url{http://www.gartner.com/newsroom/id/3165317}}, 6,4 bilhões de coisas estarão conectadas até o final de 2016 e, em 2020 esse número atingirá cerca de 20,8 bilhões. A previsão de \cite{Sundmaeker:2010}, a qual dizia que a denominada Internet dos PCs seria movida para o que se chama de Internet das Coisas fica então mais evidente neste atual cenário.

A ideia básica da \textit{Internet of things (IoT)}, traduzido para o português como Internet das Coisas é a presença pervasiva de uma variedade de "coisas ou objetos", tais como RFID tags, sensores, telefones móveis, dentre outros. Os quais, através de esquemas de endereçamento único são capazes de interagir com os outros e cooperar com seus vizinhos para alcançar um objetivo em comum \cite{Atzori:2010}.

Apesar da grande potencialidade da Internet das Coisas, a qual gerou em 2015 um faturamento em torno de 130,33 bilhões de dólares americanos e tem prospecção de chegar até 883.55 bilhões em 2020\footnotemark \footnotetext{http://www.marketsandmarkets.com/Market-Reports/iot-application-technology-market-258239167.html}, ainda existem muitos desafios a serem vencidos, tais como segurança da informação, armazenamento e processamento de grande quantidade de dados, dentre outros. Adentrando um pouco em um dos desafios da IoT, o da disponibilidade de uma interface de comunicação (acesso aos serviços e informações dos dispositivos) e programação comum aos objetos, pode-se dizer que a falta desta padronização faz com que se torne oneroso o desenvolvimento de aplicações para o objeto. Mais difícil ainda é prover uma única funcionalidade ou serviço com a composição dos diversos objetos. Para diminuir a dificuldade deste cenário, pode-se disponibilizar os dispositivos como serviços Web, desta forma pode-se utilizar os protocolos Web como linguagem comum de integração dos dispositivos a Internet \cite{Franca:2011}.

Diante deste cenário de crescente adição de dispositivos a IoT, muitos dispositivos quando de forma isolada, não conseguem atender aos requisitos dos usuários finais (humano ou outro dispositivo). Assim, há a necessidade de composição destes com outros dispositivos para prover funcionalidades que atendam a esses requisitos. Quando existe uma composição de dispositivos, o comportamento de pelo menos um destes pode provocar interação de características com efeitos colaterais indesejáveis, que é quando esta composição leva a um estado inconsistente do sistema, um sistema instável ou dados imprecisos. 

Assim, o presente trabalho propõe detectar efeitos colaterais indesejáveis entre dispositivos através da meta-classificação, metodologia que utiliza um conjunto de classificadores para construir diversas hipóteses (modelos). Então, é utilizado o conhecimento de cada modelo afim de se chegar em uma decisão final mais confiável \cite{Melville:2004}.

Os dados de treino foram gerados manualmente a partir de um conjunto de objetos do cenário ``Levar compras'', o qual foi utilizado no estudo de caso do \textit{Home Network System}.

Os resultados obtidos demonstraram que através da meta-classificação foi possível detectar os efeitos colaterais indesejáveis com mais de 90\% de precisão.

O restante deste trabalho está organizado da seguinte maneira:

\begin{itemize}
\item Capítulo \ref{ch:theory}: Traz todo embasamento teórico necessário para compreender o estudo realizado neste trabalho.
\item Capítulo \ref{**}:  
\item Capítulo \ref{**}: 
\item Capítulo \ref{**}:
\end{itemize}

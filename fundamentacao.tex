\section{Internet das Coisas}
A Internet nesta última década tem contribuído de forma significativa na economia e sociedade, deixando como legado uma notável infraestrutura de rede de comunicação. O seu maior disseminador nesse período, vem sendo \textit{WWW (World Wide Web)}, o qual permite o compartilhamento de informação e mídia de forma global\cite{Chandrakanth:2014}. 

No âmbito da economia, por exemplo, o E-Commerce permitiu potencializar as vendas de produtos e serviços, com um faturamento estimado para o ano de 2016 de aproximadamente 56,8 bilhões de reais no Brasil, segundo ecommercenews\footnotemark \footnotetext{https://ecommercenews.com.br/noticias/pesquisas-noticias/e-commerce-brasileiro-deve-crescer-18-e-faturar-r-568-bilhoes-em-2016}. Além do benefício direto para sociedade provindo do E-Commerce, onde as pessoas podem realizar pesquisa de preços, encontrar serviços diversos e, ter a comodidade de realizar tudo isto através de um dispositivo com um navegador e acesso a Internet, esta dispõe a sociedade diversas outras oportunidades, como cursos a distância oferecidos por diversas universidades de todo o mundo, a exemplo dos cursos disponibilizados pela plataforma Coursera\footnotemark \footnotetext{\url{https://pt.coursera.org/}}. 

A Internet está se tornando cada vez mais persistente no cotidiano, devido, por exemplo, ao crescente número de usuários de dispositivos móveis, os quais possuem tecnologias de conexão com a Internet, as quais cada dia tornam-se mais acessíveis (presentes em locais que não tinham e, mais baratas)\cite{Chandrakanth:2014}.

Em 2010 havia aproximadamente 1,5 bilhão de PCs conectados a Internet e mais que 1 bilhão de telefones móveis\cite{Sundmaeker:2010}. Segundo Gartner\footnotemark \footnotetext{\url{http://www.gartner.com/newsroom/id/3165317}}, 6,4 bilhões de coisas estarão conectadas até o final de 2016 e, em 2020 esse número atingirá cerca de 20,8 bilhões. A previsão de \cite{Sundmaeker:2010}, a qual dizia que a denominada Internet dos PCs seria movida para o que se chama de Internet das Coisas fica então mais evidente neste atual cenário.

A ideia básica da \textit{Internet of things (IoT)}, traduzido para o português como Internet das Coisas é a presença pervasiva de uma variedade de "coisas ou objetos", tais como RFID tags, sensores, telefones móveis, dentre outros. Os quais, através de esquemas de endereçamento único são capazes de interagir com os outros e cooperar com seus vizinhos para alcançar um objetivo em comum\cite{Atzori:2010}. Outros exemplos de "coisas ou objetos" podem ser pessoas, geladeiras, televisores, veículos, roupas, medicações, livros, passaportes, contanto que possam ser identificadas unicamente e possam se comunicar com as outras coisas.
